\section{Downstream Global Health Analysis}

\subsection*{Q1: Research question (2 pts)}
\textit{Among the papers in the project instructions, identify a research question they address that you could explore using the TweetsCOV19 dataset. For example, set out to analyze sentiments towards Covid-19 as a function of time or geographical location, or of any sub-topic related to the pandemic. Motivate the relevance of your question in the context of the pandemic. Feel free to use another peer-reviewed paper analyzing COVID tweets for public health insights for inspiration. Ensure to provide a correct citation.}

We have chosen to look at the paper “How epidemic psychology works on Twitter: evolution of responses to the COVID-19 pandemic in the U.S.“ by Aiello, L.M., Quercia, D., Zhou, K. et al. \cite{aiello2021epidemic}. We choose to look at the evolution in time of sentiment and emotion with regards to COVID-19, in the United States. 

The paper focuses on a theory called “Epidemic Psychology” by Philip Strong \cite{strong1990epidemic}, which is a sociological study of psycho-social ‘epidemics’ of fear, moralization and action that occur during major health epidemics. The authors wish to observe whether they can empirically observe these phases of emotion spreading as an epidemic through language, through the analysis of tweets, during the COVID-19 pandemic. 

Analyzing the evolution of emotion and sentiment throughout the COVID-19 pandemic through the use of tweets is a very important undertaking, as it helps inform future public health policies. Indeed, as information dissemination is extremely rapid nowadays, the spreading of fear, panic, and the amplifying misinformation, is a grave danger to public health messaging. Understanding past trends can help prevent these feelings that ultimately have a big impact on policy. Finally, we also choose to focus on the US as per the paper, as the US has the highest proportion of Twitter users, and because it is easier to relate evolution of emotions in time to the events of a single country.


\subsection*{Q2: Method choice and design (5 pts).}
\textit{Among the methods you have explored in Part 2, select one approach to tackle this research question using the TweetsCOV19 dataset and motivate your choice. Detail any necessary modifications you implement to achieve this analysis and performance metrics to measure the success of your approach. Provide a code snippet used to perform this analysis.}
TODO @Juraj

\subsection*{Q3: Results \& Analysis (6 pts).}
\textit{ Analyze your results and provide numerical evaluation and visualizations showcasing your findings. (3 pts) Explain what conclusions can be drawn from these, as well as key takeaways which answer (or partially answer) your research question. (3 pts)}
TODO @Juraj


\subsection*{Q4: Comparison to literature (3 pts).}
\textit{ Discuss whether your results support or disagree with the paper you have chosen for inspiration. Provide plausible reasons for different findings or any performance discrepancies.}
TODO @Juraj


\subsection*{Q5: Discussion (3 pts).}
\textit{ Discuss the pros and cons of your approach compared to the one used in the paper.}
TODO @Juraj


\subsection*{Q6: Summary \& Conclusion (1 pt).}
\textit{ Provide a summary of your analysis and the insights it provides about your research question.}
TODO @Juraj
